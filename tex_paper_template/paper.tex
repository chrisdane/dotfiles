\documentclass[11pt]{article} % default font size: 10pt 
\usepackage[margin=1in]{geometry} % 0.7in
\usepackage[parfill]{parskip}
%\usepackage[utf8]{inputenc} % utf8 or latin1; default: utf8 
\usepackage[english]{babel} % USenglish, english or german
\usepackage{float} % for [H] position
\usepackage{lmodern}
\usepackage{setspace} % \singlespacing, \onehalfspacing, \doublespacing
\doublespacing
\usepackage{amsmath,amssymb,amsfonts,amsthm}
\usepackage{array}
\usepackage{graphicx}
\usepackage{longtable}
\usepackage{booktabs}
\usepackage{multirow}
\usepackage{makecell} % \thead
% overwrite \thead for nicer vertical orientation in multi-line table headers:
\newcommand*{\mythead}[1]{\multicolumn{1}{c}{\bfseries\begin{tabular}{@{}c@{}}#1\end{tabular}}}
\newcommand*{\mytheadl}[1]{\multicolumn{1}{l}{\bfseries\begin{tabular}{@{}c@{}}#1\end{tabular}}}
\usepackage{caption} % \captionsetup
\usepackage[pdfpagelabels,plainpages=false,hypertexnames=false]{hyperref}
\hypersetup{
    colorlinks = true,
    allcolors = blue % all links same color
    %linkcolor = red
    %anchorcolor = black
    %citecolor = green
    %filecolor = cyan
    %menucolor = red
    %runcolor = cyan % same as file color
    %urlcolor = magenta
    %allbordercolors = blue % all border same color
    %citebordercolor = rgb 0 1 0
    %filebordercolor = rgb 0 .5 .5
    %linkbordercolor = rgb 1 0 0
    %menubordercolor = rgb 1 0 0
    %urlbordercolor = rgb 0 1 1
    %runbordercolor = rgb 0 .7 .7
    %breaklinks = true
    %hidelinks = true % good for latexdiff result
}
\usepackage{lineno}
\linenumbers
\usepackage[round]{natbib}

\begin{document}

\thispagestyle{empty} % first page without page number
\begin{center}

\vspace*{2cm}
\LARGE{\textbf{Title}}\\
\vspace{2cm}

\normalsize
My Name

\end{center}
\vfill
\textit{Corresponding author}: My Name, \href{mailto:me.de}{myname@myprovider.org}

\newpage
\pagenumbering{arabic} % page numbers for rest of the document

\section*{Abstract}
\addcontentsline{toc}{section}{Abstract}

This is the abstract.

\section{Introduction}

Fig. \ref{fig:a} is the A-figure by \citet{einstein}.

\begin{figure}[H]
\centering\noindent\includegraphics[width=0.25\textwidth]{example-image-a}
\caption{Example A-figure.}
\label{fig:a}
\end{figure}

Fig. \ref{fig:b} is the B-figure \citep{einstein}.

\begin{figure}[H]
\centering\noindent\includegraphics[width=0.25\textwidth]{example-image-b}
\caption{Example B-figure.}
\label{fig:b}
\end{figure}

\subsection{This is a section title with an equation in it; namely \texorpdfstring{$\delta^{18}$}{d}O}

Fig. \ref{fig:a2} is the A-figure again.

\begin{figure}[H]
\centering\noindent\includegraphics[width=0.25\textwidth]{example-image-a}
\caption{Example A-figure again.}
\label{fig:a2}
\end{figure}

Tab. \ref{tab:table} is a table example.

\begin{table}[H]
\centering
\caption{This is the caption of a table example.}
\label{tab:table}
\begin{tabular}{llr}
\toprule
\multicolumn{2}{c}{Item}\\
\cmidrule(r){1-2}
Animal & Description & Price (\$)\\
\midrule
Gnat & per gram & 13.65\\
& each & 0.01\\
Gnu & stuffed & 92.50\\
Emu & stuffed & 33.33\\
Armadillo & frozen & 8.99\\
\bottomrule
\end{tabular}
\end{table}

\section{Results}

Tab. \ref{tab:longtable_default} is a longtable example.

\begin{longtable}{lll}
\caption{This is the caption of a longtable example.}\label{tab:longtable_default}\\
\toprule
Animal & Description & Price (\$)\\
\midrule
Gnat & per gram & 13.65\\
Gnu & stuffed & 92.50\\
Emu & stuffed & 33.33\\
Armadillo & frozen & 8.99\\
\bottomrule
\end{longtable}

\bibliography{paper} % also creates the section header ``References''
\bibliographystyle{ametsoc2014}
\addcontentsline{toc}{section}{References}

\section*{Appendix A}\addcontentsline{toc}{section}{Appendix A}
\renewcommand\thefigure{A\arabic{figure}}
\setcounter{figure}{0}
\renewcommand{\thetable}{A\arabic{table}}
\setcounter{table}{0}

Fig. \ref{fig:appendix1} is an appendix figure with a specific appendix prefix.

\begin{figure}[H]
\centering\noindent\includegraphics[width=0.25\textwidth]{example-image-a}
\caption{Example figure with a specific appendix prefix.}
\label{fig:appendix1}
\end{figure}

Tab. \ref{tab:longtable_reduced_row_height} is a longtable example with reduced row height and specific appendix prefix.

\begin{longtable}{lll}
\caption{This is the caption of a longtable with reduced row height example with a specifix appendix prefix.}\label{tab:longtable_reduced_row_height}\\
\begin{footnotesize}
\bgroup
\def\arraystretch{0.8} % change rowsize; 1=default
\begin{tabular}{lll}
\toprule
Animal & Description & Price (\$)\\
\midrule
Gnat & per gram & 13.65\\
Gnu & stuffed & 92.50\\
Emu & stuffed & 33.33\\
Armadillo & frozen & 8.99\\
\bottomrule
\end{tabular}
\egroup
\end{footnotesize}
\end{longtable}

Tab. \ref{tab:longtable_reduced_row_height_and_linebreaks} is a longtable example with reduced row height and line breaks in the table headers and a specific appendix prefix.

\fontsize{9}{6}\selectfont
\begin{longtable}{@{}rcrrcc@{}}
\caption{This is the caption of a longtable with reduced row height and multi-row table headers example with a specifix appendix prefix.}\label{tab:longtable_reduced_row_height_and_linebreaks}\\
\toprule
\thead{ID} & \thead{Database name} & \thead{Size\\(MB)} & \thead{No. of\\records} & \thead{Time stamp\\1st record} & \thead{Time stamp\\last record}\\
\midrule
1 & dummie & 2.1 & 33 & dummie & dummie\\
2 & dummie & 4.3 & 67 & dummie & dummie\\
\bottomrule
\end{longtable}
\normalsize

Text default size after the longtable with the reduced font size.

Tab. \ref{tab:longtable_reduced_row_height_and_linebreaks2} is another longtable example with reduced row height and line breaks in the table headers and a specific appendix prefix.

\captionsetup{width=\textwidth}
\fontsize{9}{6}\selectfont
\begin{longtable}{@{}rlrrrrrr@{}}
\caption{This is the caption of a longtable with reduced row height and multi-row table headers example with a specifix appendix prefix. Note the escape of the [ symbol with \{[\} in the 5th header. This is necessary since the linebreak shall be just before the [ symbol. \href{www.example.com}{This is a href in a caption.}}\label{tab:longtable_reduced_row_height_and_linebreaks2}\\
\toprule
\thead{No} & \thead{LiPD ref} & \thead{lon [$^{\circ}$]} & \thead{lat [$^{\circ}$]} & \thead{Start} & \thead{End} & \thead{P\\trend} & \thead{trend\\summary}\\
\midrule
1 & \href{http://lipdverse.org/globalHolocene/1_0_0/31Lake.Eisner.1995.html}{31Lake.Eisner.1995} & -50.47 & 67.05 & 5769 & 667 & 408 & r=0.64, p=1e-04, df=30\\
2 & \href{http://lipdverse.org/globalHolocene/1_0_0/AgeroedsMosse.Nilsson.1964.html}{AgeroedsMosse.Nilsson.1964} & 13.42 & 55.83 & 6910 & 19 & -711 & r=0.79, p<1e-5, df=74\\
3 & \href{http://lipdverse.org/globalHolocene/1_0_0/AlpidiRobieiValBavona.Welten.1982.html}{AlpidiRobieiValBavona.Welten.1982} & 8.52 & 46.45 & 6826 & 63 & 244 & r=0.73, p=1e-04, df=21\\
4 & \href{http://lipdverse.org/globalHolocene/1_0_0/Altenweiher.DeValk.1981.html}{Altenweiher.DeValk.1981} & 6.99 & 48.01 & 6970 & 139 & 185 & r=0.47, p=2e-04, df=55\\
5 & \href{http://lipdverse.org/globalHolocene/1_0_0/BeaufortBirkenbach.EPD.html}{BeaufortBirkenbach.EPD} & 6.13 & 49.85 & 3702 & 5 & 158 & r=0.48, p=3e-04, df=50\\
6 & \href{http://lipdverse.org/globalHolocene/1_0_0/BebrukasLake.Shulija.1967.html}{BebrukasLake.Shulija.1967} & 24.12 & 54.09 & 6721 & 286 & 249 & r=0.72, p<1e-5, df=28\\
7 & \href{http://lipdverse.org/globalHolocene/1_0_0/Besbog.EPD.html}{Besbog.EPD} & 23.67 & 41.75 & 6904 & 169 & 474 & r=0.81, p<1e-5, df=19\\
8 & \href{http://lipdverse.org/globalHolocene/1_0_0/BledowoLake.Binka.1988.html}{BledowoLake.Binka.1988} & 20.67 & 52.55 & 6976 & 157 & -83 & r=0.46, p<1e-5, df=144\\
9 & \href{http://lipdverse.org/globalHolocene/1_0_0/BoehnigseeGoldmoos.Markgraf.1969.html}{BoehnigseeGoldmoos.Markgraf.1969} & 7.84 & 46.26 & 6967 & 248 & 178 & r=0.57, p=2e-04, df=36\\
10 & \href{http://lipdverse.org/globalHolocene/1_0_0/Breitnau-Neuhof.Roesch.2009.html}{Breitnau-Neuhof.Roesch.2009} & 8.07 & 47.93 & 6915 & 199 & 142 & r=0.43, p=0.004, df=42\\
\bottomrule
\end{longtable}
\normalsize

Text default size after the longtable with the reduced font size.

\end{document}
